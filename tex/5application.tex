% !TeX root = ../thesis.tex

\chapter{Application to FCh model}\label{chap:model}

\section{Flexible Chain with Helical Interactions (FCh)}
The molecular model is given in Fig. \ref{fig:fch},
\begin{figure}[H]
 	\centering
 	\includegraphics[width=\linewidth]{fch.png}
	\caption[Schematic of FCh model]{Schematic figure of FCh model\cite{Liang2019PRE}}
	\label{fig:fch}
\end{figure}

There are several defining parameters of a FCh molecule $(m,\phi_0)$, the former is the length of FCh and the latter is its helix internal angle. In addition, to identify a FCh molecule in the space, it suffices to use $(q,u,\phi)$, in which $q$ is the position of center of mass, $u$ is the orientation, and $\phi$ is the internal azimuthal angle.

Denote by $\sigma$ the length unit, $\epsilon$ the energy unit, $\sigma_b = 2^{1/6}\sigma$ and $\sigma_i = 0.1\sigma_b$ are the diameter of the backbone bead and interaction site bead. And there are two kinds of interactions involved, soft repulsion between backbone beads by Weeks-Chandler-Anderson (WCA) potential, short-ranged attractions between interaction beads by shifted Lennard-Jones (sLJ) potential truncated at $r_c=2.5\sigma$.

\begin{equation}
	U_{\mathrm{WCA}}(r)=\left\{
	\begin{array}{ll}
		4 \epsilon\left[\left(\frac{\sigma_b}{r}\right)^{12}-\left(\frac{\sigma_b}{r}\right)^{6}\right]+\epsilon & r<\sigma_b \\ 0 & r \geq \sigma_b
	\end{array}\right.
\end{equation}

\begin{equation}
	U_{\mathrm{sLJ}}(r)=\left\{
	\begin{array}{ll}
		4 \epsilon\left[\left(\frac{\sigma_i}{r}\right)^{12}-\left(\frac{\sigma_i}{r}\right)^{6}\right]-U_{\mathrm{LJ}}\left(r_c\right) & r<2.5 \sigma\\0 & r \geq 2.5 \sigma
	\end{array}\right.
\end{equation}

As for the potential between backbone and interaction beads, it is sLJ potential with parameters are derived by Lorentz-Berthelot mixing rule in the previous simulation work, but here it is ignored, there only b-b interaction and i-i interaction in this theoretical work.

\section{Existing Simulation Results}\label{Sec:SimRes}
Molecular dynamics simulations were performed in the paper and it is found there are three existing phases, isotopic, nematic and cholesteric. To illustrate further, take the example in case of $m=10, \phi_0=30^\circ$, the pressure and phase diagram is shown below.

\begin{figure}[H]
 	\centering
 	\includegraphics[width=\linewidth]{phaseDiag.png}
	\caption[Phase diagram of FCh using MD simulation]{Phase Diagram of FCh using MD simulation\cite{Liang2019PRE}}
	\label{fig:MDdiag}
\end{figure}

Note in the diagram, $P^*=P\sigma^3/\epsilon$. And in this figure we know the transition pressure and transition packing fraction are $P_0^*=0.1v_m=1.332, \eta_0=0.24$

\section{DFT of FCh Model}
This part will make use of all the methods introduced before and apply to FCh model. In particular, we will illustrate how to use variational method and iteration to find out ODF which corresponds to the minimal free energy.

Recall that the free energy is unpacked into two parts,
\begin{equation}
	F^*=F_{id}^*+F_{ex}^*,
\end{equation}
the first term contains both free energy of ideal gas and the entropy contributed by ODF,
\begin{equation}
	F_{id}^*=\ln\eta-1+\int\on{d}uf(u)\ln f(u),
\end{equation}
in fact the first term is $\ln\rho\Lambda^3$, but this does not matter because it only differs by a constant with $\ln\eta$.

Then due to Straley\cite{Straley1973Frank, Straley1976Theory} , the excess free energy is written in the form
\begin{align}
	F_{ex}^*&=\frac{\eta}{2}\int\on{d}u_1\on{d}u_2M_0^*(u_1,u_2)\varphi(u_1)\varphi(u_2)-K_t^*q+\frac{1}{2}K_2^*q^2\\
	K_t^*&=-\frac{\eta}{2}\int\on{d}u_1\on{d}u_2M_1^*(u_1,u_2)\varphi(u_1)\varphi'(u_2)u_{2y}\\
	K_2^*&=-\frac{\eta}{2}\int\on{d}u_1\on{d}u_2M_2^*(u_1,u_2)\varphi'(u_1)\varphi'(u_2)u_{1y}u_{2y}
\end{align}
and in which $M_k$ has been defined above.

In addition, the first term of $F_{ex}^*$ can be further divided into repulsion part and attractive part, the repulsion part can then be approximated by Parsons-Lee's method,
\begin{equation}
	\frac{\eta}{2}\int\on{d}u_1\on{d}u_2M_0^*(u_1,u_2)\varphi(u_1)\varphi(u_2)=G(\eta)V_{int}+\frac{\eta}{2}A_{int}.
\end{equation}

in which $V_{int}$ and $A_{int}$ is the abbreviation for two integrals, they will be used further in the calculation of other physical properties,
\begin{equation}\label{Eqn:VAstar}
	\begin{split}
		V^*(u_1,u_2)=\frac{1}{v_m}\int\on{d}q\left[1-e^{-\beta U_{rep}(0,u_1;q,u_2)}\right],\\
		V_{int} = \int\on{d}u_1\on{d}u_2V^*(u_1,u_2)\varphi(u_1)\varphi(u_2), \\
		A^*(u_1,u_2)=\frac{1}{v_m}\int\on{d}q\left[1-e^{-\beta U_{att}(0,u_1;q,u_2)}\right],\\
		A_{int}=\int\on{d}u_1\on{d}u_2A^*(u_1,u_2)\varphi(u_1)\varphi(u_2).
	\end{split}
\end{equation}

According to Sec. \ref{Sec:Joachim}, the expression for $V^*$ can be approximated directly by Eqn. \ref{Eqn:Joachim}, since the repulsion interaction in FCh model is nearly hard.

Up to now all the required quantities are presented. In the next section we write down explicitly all the important physical properties.

\section{Properties}
All the properties calculated in this part should be calculated after computing ODF using iteration method, but for logical coherence, we derive them first.

Remark: keep in mind that $F^*\equiv\beta F/N$.
\subsection{Reduced free energy}
The general expression of reduced free energy can be collected into one expression
\begin{equation}\label{Eqn:FreeEne}
	F^*=\ln\eta -1+\int\on{d}u\varphi(u)\ln \varphi(u)+\frac{4\eta-3\eta^2}{8(1-\eta)^2}V_{int}+\frac{\eta}{2}A_{int}-K_t^*q+\frac{1}{2}K_2^*q^2.
\end{equation}
To be more precise, the expression above is the free energy of cholesteric phase, as for the free energy of nematic phase, the last two terms are dropped
\begin{equation}\label{Eqn:EneNem}
	F_{n}^*=\ln\eta -1+\int\on{d}u\varphi(u)\ln \varphi(u)+\frac{4\eta-3\eta^2}{8(1-\eta)^2}V_{int}+\frac{\eta}{2}A_{int}.
\end{equation}
And as for the free energy of isotropic phase, it suffices to let $\varphi(u)\equiv \frac{1}{4\pi}$ and substitute into Eqn. \ref{Eqn:EneNem}.

\subsection{Reduced Pressure}
The reduced pressure $P^*=\beta v_m P=-\beta v_m \dfrac{\partial F}{\partial V}= \eta^2\dfrac{\partial F^*}{\partial\eta}$
\begin{equation}\label{Eqn:Press}
	P^* = \eta + \frac{\eta^2(2-\eta)}{4(1-\eta)^3}V_{int}+\frac{1}{2}\eta^2A_{int}-\frac{1}{2}\eta K_t^*q
\end{equation}

\subsection{Chemical Potential}
The reduced chemical potential $\mu^*=\beta\mu=\beta\dfrac{\partial F}{\partial N}=\eta\dfrac{\partial F^*}{\partial\eta}+F^*$
\begin{equation}\label{Eqn:ChemPot}
	\mu^*=\ln\eta+\int\on{d}u\varphi(u)\ln \varphi(u)+\frac{\eta(3\eta^2-9\eta+8)}{8(1-\eta)^3}V_{int}+\eta A_{int}-K_t^*q.
\end{equation}

\subsection{Order Parameter}
The order parameter is standard from Onsager's theory
\begin{equation}
	S_2=\int \on{d}u\varphi(u)\left[\frac{1}{2}(3\cos^2\theta -1)\right],
\end{equation}
in which $\theta$ is the angle between $u$ and $e_3$, precisely given by $\cos\theta = u_z = u\cdot e_3$.

\section{Code Calibration: Hard Spherocylinders (HSC) Model}
To verify the correctness of the computation program, we used the data from Jackson's work\cite{Jackson1996JCP} in 1996. In the paper the author re-examined the phase diagram of hard spherocylinders by Monte-Carlo simulation in a canonical ensemble. The schematic of hard spherocylinders is given below. We used our theoretical method to calculate the phase diagram of HSC, and the result is in well accordance with Jackson's simulation results.

\begin{figure}[H]
 	\centering
 	\includegraphics[width=\linewidth]{HSC.png}
	\caption[Schematic of Hard Spherocylinders]{Schematic of Hard Spherocylinders\cite{Jackson1996JCP}}
	\label{fig:MDdiag}
\end{figure}

Jackson et al. calculated HSC with four different ratio of length and diameter $\kappa=L/D=3,3.2,4,5$, and the nematic-isotropic transition point is compared in the next table.

\begin{table}[H]
	\caption{Comparison of transition point}
	\label{tab:comparison1995}
	\centering
	\begin{tabular}{c|cccc}
		\toprule
		 & $\kappa=3$ & $\kappa=3.2$ & $\kappa=4$ & $\kappa=5$\\
		\midrule
		$\eta_{MC}$ &  $0.577$ & $0.513$ & $0.472$ & $0.408$\\
		$\eta_{DFT}$ & $0.530$ & $0.515$ & $0.460$ & $0.405$\\
		\bottomrule
	\end{tabular}
\end{table}

It can be seen from Tab. \ref{tab:comparison1995} that the result given by DFT method is in accordance with the result given by Monte Carlo method in a canonical ensemble. The more detailed data of pressure is also in well accordance.


\section{Program Development}
To calculate all the properties and phase diagram mentioned above, a whole program in C++ has been implemented from scratch and been uploaded on \href{https://github.com/WolframC/liquidcrystal}{Github}. The program is transferrable for different kind of liquid crystals as long as the class which defined the molecule structure is added.

Overall, the program contains three main features
\begin{enumerate}
	\item Monte Carlo computation of ensemble average of quantities in terms of energy.
	\item Computation of ODF by iteration method.
	\item Calculation of free energy and remaining physical properties.
\end{enumerate}

And this section serve as both a report for what has been done and instructions for those who need to continue the liquid crystal project study. We first recapitulate the expressions we are going to compute, and further in following chapters we illustrate concrete implementation involved.

\subsection{Monte Carlo Method}
According to the theory introduced before, we need to first calculate properties which are irrelevant to ODF, namely, they are $A^*, V^*, M_k^*$ (which are given by Eqn. \ref{Eqn:VAstar}, Eqn. \ref{Eqn:Mk}). To see how they are calculated, $V^*$ has been given by Joachim's approximation in Eqn. \ref{Eqn:Joachim}. And $A^*, M_k^*$ are computed using Monte Carlo method.
\begin{equation*}	
	\begin{split}
		V^*(u_1,u_2)=11-\frac{3}{m}&+3.5339(m+\frac{1}{m}-2)\sin\theta,  \cos\theta=u_1\cdot u_2,\\
		A^*(u_1,u_2)&=\frac{1}{v_m}\int\on{d}q\left[1-e^{-\beta U_{att}(0,u_1;q,u_2)}\right],\\
		M_k^*(u_1,u_2)&=\frac{1}{v_m}\int\on{d}q \left[1-e^{-\beta u(0,u_1;q,u_2)}\right]q_z^k.
	\end{split}
\end{equation*}

Note that in fact to identify a FCh model, it is necessary to designate at least three parameters: $(q,u,\phi)$, in which $q$ denotes the coordinate of its center of mass, $u$ denotes its orientation, and $\phi$ denotes the internal azimuthal angle. So the integral in $A^*,M_k^*$ has to be slightly modified and made more precise:

\begin{equation}
	\begin{split}
		A^*(u_1,u_2)&=\frac{1}{v_m}\int_V\on{d}q \int_0^{2\pi}\on{d}\phi_1\int_0^{2\pi}\on{d}\phi_2 \left[1-e^{-\beta U_{att}(0,u_1,\phi_1;q,u_2,\phi_2)}\right],\\
		M_k^*(u_1,u_2)&=\frac{1}{v_m}\int_V\on{d}q \int_0^{2\pi}\on{d}\phi_1\int_0^{2\pi}\on{d}\phi_2 \left[1-e^{-\beta u(0,u_1, \phi_1; q,u_2, \phi_2)}\right]q_z^k.
	\end{split}
\end{equation}

In the expression $V$ is taken to be the whole space in which MC process will take place, and let the length scale on one dimension to be two times of a single molecule, i.e. denote $L = 2m\sigma_b$.

Since the two integrals are of the same type, we only illustrate the computation for $M_k^*$. By using cylindrical coordinates, write $\on{d}q=r\on{d}r\on{d}z\on{d}\phi$. Then the expression of $M_k^*$ becomes
\begin{equation}\label{Eqn:IntegMk}
	M_k^*(u_1,u_2)=\frac{2\pi}{v_m}\int_{-L/2}^{L/2}\on{d}z\int_{-L/2}^{L/2}\on{d}r \int_0^{2\pi}\on{d}\phi_1\int_0^{2\pi}\on{d}\phi_2 \left[1-e^{-\beta u(0,u_1, \phi_1; q,u_2, \phi_2)}\right]z^kr.
\end{equation}

Except for using Monte Carlo method to estimate the preceding integral, we can also use an averaged method by dividing $z,r,\phi_1,\phi_2$ into uniform pieces. In practice, $z,r$ are always taken to be divided into $50$ pieces and $\phi_1,\phi_2$ are divided into $10$ pieces.

\subsubsection{Collision Detection}
In the estimation of the integral introduced above, there are many pairs of configurations which made $u(0,u_1,\phi_1;q,u_2,\phi_2)=0$ and hence contributes nothing to the whole integral, since the interaction between FCh molecules are relatively short-ranged. So we can calculate the shortest distance between two molecules and judge if it is within the cutoff distance to optimize the integral computation. We give the algorithm here without proof, since it is easy with elementary vector analysis.

\begin{algorithm}[H]
	\caption{Shortest Distance between a Point and a Segment}
	\label{Alg:DisPointSeg}
	\KwData{point $\vec{p}$, start point of the segment $\vec{s}$, end point of the segment $\vec{t}$}
	\KwResult{shortest distance $d$ between $\vec{p}$ and the segment defined by $\vec{s},\vec{t}$}
	$\vec{u}\leftarrow\dfrac{\vec{t}-\vec{s}}{\|\vec{t}-\vec{s}\|_2}$\;
	$\vec{ps}\leftarrow\vec{p}-\vec{s}$, $\vec{pt}\leftarrow\vec{p}-\vec{t}$\;
	\uIf{$\vec{u}\cdot\vec{ps}<0$}{
		$d\leftarrow\|\vec{ps}\|_2$\;
	}\uElseIf{$\vec{u}\cdot\vec{pt}>0$}{
		$d\leftarrow\|\vec{pt}\|_2$\;
	}\Else{
		$d\leftarrow\|\vec{u}\times\vec{ps}\|_2$\;
	}
\end{algorithm}

\begin{algorithm}[H]
	\caption{Shortest Distance between Two Segments}
	\label{Alg:DisSegSeg}
	\KwData{length, center and unit orientation vector of the segment $l_i, \vec{r}_i, \vec{u}_i$}
	\KwResult{shortest distance $d$ between the two segment}
	$\vec{s}\leftarrow\vec{r}_1-\vec{r}_2$\;
	$f\leftarrow$True\;
	\uIf{$1-(\vec{u}_1\cdot\vec{u}_2)^2=0$}{
		$f\leftarrow$False\tcp*[l]{Parallel segments}
	}\Else{
		$\lambda_1\leftarrow\dfrac{1}{1-(\vec{u}_1\cdot\vec{u}_2)^2}\left(-\vec{s}\cdot\vec{u}_1+(\vec{s}\cdot\vec{u}_2)(\vec{u}_1\cdot\vec{u}_2)\right)$\;
		$\lambda_2\leftarrow\dfrac{1}{1-(\vec{u}_1\cdot\vec{u}_2)^2}\left(\vec{s}\cdot\vec{u}_2-(\vec{s}\cdot\vec{u}_1)(\vec{u}_1\cdot\vec{u}_2)\right)$\;
		
		\eIf{$\left(|\lambda_1| > l_1/2\right) \vee \left(|\lambda_2|>l_2/2\right)$}{
			$f\leftarrow$False\tcp*[l]{Skew segments}
		}{
			$f\leftarrow$True\tcp*[l]{Segments intesects}
		}
	}
	
	\eIf{$f$}{
		$d\leftarrow\|\vec{s}+\lambda_1\vec{u}_1-\lambda_2\vec{u}_2\|_2$
	}{
		$d$ is taken to be the minimum of four shortest distance from four endpoints to the other segments using Alg. \ref{Alg:DisPointSeg}.
	}
\end{algorithm}

Using the algorithms introduced above, the efficiency of Monte Carlo estimation of the integral can be largely improved.

\subsection{DFT calculation}
With all the data coming from the Monte Carlo method, using iteration to find out the corresponding ODF with a given packing fraction $\eta$ can be directly performed, which has been already illustrated in Sec. \ref{Sec:Iter}. But to clarify here, we must point out the concrete discretization method of a function on the sphere. we divide every longitude line into $20$ parts, and latitude line into $40$ part (except at two poles), so this gives $40\times 19+2 = 762$ point on the sphere. Hence the ODF is actually a $762$-dimensional vector in computation code. So $\on{d}\phi=\on{d}\theta =\pi/20$And $\on{d}u$ actually becomes $\dfrac{\pi^2}{400}\sin\phi$, with $u\cdot e_3=\cos\phi$.

\section{Iteration Process}

The DFT process should give a convergent result of ODF, we put an example here showing this.
\begin{figure}[H]
 	\centering
 	\includegraphics[width=0.8\linewidth]{snf.jpg} \\
	\caption[Step of DFT process]{Step of DFT process}
	\label{fig:DFTstep}
\end{figure}
From this we can see that near the first-order transition point, the iteration step reaches a peak, which means it the state is very non stable near transition and may jump between two different states.

\section{The Isotropic-Nematic Phase Transition}
The isotropic-Nematic phase transition is observed using this theoretical method, and a typical result of physical properties at $m=10, \phi_0=30^\circ$ is given below. A sudden change in the order parameter implies an isotropic-nematic phase transition, and we can also see that the free energy of nematic phase is lower than free energy of isotropic phase.
\begin{figure}[H]
	\begin{minipage}{0.5\textwidth}
		\centering
		\includegraphics[width=\linewidth]{orderparam.pdf}
		\caption{Relation between order parameter and packing fraction}
		\label{fig:orderParam}
	\end{minipage}\hfill
	\begin{minipage}{0.5\textwidth}
 		\centering
 		\includegraphics[width=\linewidth]{freeEne.pdf}
 		\caption{Free energy of isotropic and nematic phases}
 		\label{fig:freeEne}
 	\end{minipage}
\end{figure}

And the sudden change at $\eta=0.268$ can also be seen in the diagram of pressure and chemical potential, there is a sudden decline as well as the case in free energy. This is the typical characteristic of a first-order transition.
\begin{figure}[H]
	\begin{minipage}{0.5\textwidth}
		\centering
		\includegraphics[width=\linewidth]{1030chemPot.pdf}
		\caption{Chemical potential of isotropic and nematic phase}
		\label{fig:chemPot}
	\end{minipage}\hfill
	\begin{minipage}{0.5\textwidth}
 		\centering
 		\includegraphics[width=\linewidth]{1030press.pdf}
 		\caption{Pressure of isotropic and nematic phase}
 		\label{fig:press}
 	\end{minipage}
\end{figure}

In addition, we compared the transition point at different length and internal azimuthal angle, the following are the data of transition point for FCh model of length $m=10,15,20$ and internal azimuthal angle $\phi_0=15^\circ, 30^\circ, 45^\circ, 60^\circ$.

\begin{figure}[H]
 	\centering
 	\includegraphics[width=0.8\linewidth]{transition.pdf} \\
	\caption[Schematic of coordinates]{Transition point of FCh}
	\label{fig:tran}
\end{figure}

It can be seen from the figure that $45^\circ$ might be a preferred internal azimuthal angle at which two adjacent FCh molecules have the strongest mutual interaction, making it more easily to undergo phase transition. As for a given internal azimuthal angel, the longer the molecule is, the larger its orientational entropy contributes to the free energy, since phase transition is in fact a balance between energy and entropy, the larger contribution from orientational entropy will result in smaller packing fraction.

\section{The Cholesteric Pitch}
The calculation for cholesteric pitch is still not clear since the computation of $K_t^*$ always gives a zero. It is very important here to note when doing integration for $M_k^*$, we need to make sure that $M_1(u_1,u_2)\neq M_1(u_2,u_1)$, otherwise $K_t^*$ must be zero. This can be seen from the following derivation by writing $\on{d}u=\sin\phi\on{d}\phi\on{d}\theta$
\begin{equation}
	\begin{split}
		K_t^*&=-\frac{\eta}{2}\int\on{d}u_1\on{d}u_2M_1^*(u_1,u_2)\varphi(u_1)\varphi'(u_2)u_{2y}\\
		&=-\frac{\eta}{2}\int_{-\pi/2}^{\pi/2}\on{d}\phi_1\varphi(\phi_1)\sin\phi_1\int_{-\pi/2}^{\pi/2}\on{d}\phi_2\varphi'(\phi_2)\sin^2\phi_2\int_{-\pi}^{\pi}\on{d}\theta_2\sin\theta_2\int_{-\pi}^{\pi}\on{d}\theta_1M_1^*(\phi_1,\theta_1;\phi_2,\theta_2).
	\end{split}
\end{equation}

Suppose $M_1^*(u_1,u_2)=M_1^*(u_2,u_1)$, then $M_1^*(\phi_1,\theta_1;\phi_2,\theta_2)=M_1^*(\phi_2,\theta_2;\phi_1,\theta_1)$, which then implies $T(\theta_2,\phi_1,\phi_2)=\int_{-\pi}^{\pi}\on{d}\theta_1M_1^*(\phi_1,\theta_1;\phi_2,\theta_2)$ is an even function with respect to all three coordinates. In addition, $\varphi(\phi_1)\sin\phi_1$ is odd in $\phi_1$, $\varphi'(\phi_2)\sin^2\phi_2$ is odd in $\phi_2$ and $\sin\theta_2$ is odd in $\theta_2$. We conclude that
\begin{equation}
	K_t^*=-\frac{\eta}{2}\int_{-\pi/2}^{\pi/2}\on{d}\phi_1\varphi(\phi_1)\sin\phi_1\int_{-\pi/2}^{\pi/2}\on{d}\phi_2\varphi'(\phi_2)\sin^2\phi_2\int_{-\pi}^{\pi}\on{d}\theta_2\sin\theta_2T(\theta_2, \phi_1,\phi_2)=0
\end{equation}

So there is something tricky in the Eqn. \ref{Eqn:IntegMk}, if we write
\begin{equation}
	E(u_1,u_2)=\int_0^{2\pi}\on{d}\phi_1\int_0^{2\pi}\on{d}\phi_2 \left[1-e^{-\beta u(0,u_1, \phi_1; q,u_2, \phi_2)}\right],
\end{equation}
Then it is worth thinking how to use a strategy in the Monte Carlo process to guarantee that $E(u_1,u_2)\neq E(u_2,u_1)$.

Up to now the calculation of $K_t^*$ always gives zero, so here we'll present some results on the scaling relation between $S$ and $K_2^*$ near the transition point.

Below is a typical figure of the scaling relation between $K_2^*$ and $S$ for $m=10,\phi_0=30^\circ$,
\begin{figure}[H]
 	\centering
 	\includegraphics[width=0.8\linewidth]{1030.pdf} \\
	\caption[Scaling relation]{Scaling Relation}
	\label{fig:scaling}
\end{figure}
The slope is $3.02$, which implies $K_2^*\sim S^3$ at moderate nematic order.

\section{Comparison with Simulation Results}
Comparing with the simulation results in Sec. \ref{Sec:SimRes}, we can see if our theory gives a very close result with the simulations on the isotropic-nematic transition point.
\begin{table}[H]
	\caption{Comparison with Simulation}
	\label{tab:compsim}
	\centering
	\begin{tabular}{c|cc}
		\toprule
		Transition Point & Pressure $P_0^*$ & Packing Fraction $\eta_0$ \\
		\midrule
		Theory &  $1.451$ & $0.269$\\
		Simulation & $1.332$ & $0.240$\\
		\bottomrule
	\end{tabular}
\end{table}

From Tab. \ref{tab:compsim} we can see that both transition pressure and packing fraction are very close. The transition pressure and packing is a bit larger than that of the simulation. This is due to the assumption that the two-body radial distribution function to be $1$. If the correct RDF is calculated, particles that and lean on each other at a short distance and on a preferred angle will contribute more to the free energy calculation, which will lower the transition point. This is the motivation of our future work.




