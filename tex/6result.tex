% !TeX root = ../thesis.tex

\chapter{Calculation Details}

\section{Program Development}
To calculate all the properties and phase diagram mentioned above, a whole program in C++ has been implemented from scratch. The program is transferrable for different kind of liquid crystals as long as the class which defined the molecule structure is added.

Overall, the program contains three main features
\begin{enumerate}
	\item Monte Carlo computation of ensemble average of quantities in terms of energy.
	\item Computation of ODF by iteration method.
	\item Calculation of free energy and remaining physical properties.
\end{enumerate}

And this section serve as both a report for what has been done and instructions for those who need to continue the liquid crystal project study. We first recapitulate the expressions we are going to compute, and further in following chapters we illustrate concrete implementation involved.

\subsection{Monte Carlo Method}
According to the theory introduced before, we need to first calculate properties which are irrelevant to ODF, namely, they are $A^*, V^*, M_k^*$ (which are given by Eqn. \ref{Eqn:VAstar}, Eqn. \ref{Eqn:Mk}). To see how they are calculated, $V^*$ has been given by Joachim's approximation in Eqn. \ref{Eqn:Joachim}. And $A^*, M_k^*$ are computed using Monte Carlo method.
\begin{equation*}	
	\begin{split}
		V^*(u_1,u_2)=11-\frac{3}{m}&+3.5339(m+\frac{1}{m}-2)\sin\theta,  \cos\theta=u_1\cdot u_2,\\
		A^*(u_1,u_2)&=\frac{1}{v_m}\int\on{d}q\left[1-e^{-\beta U_{att}(0,u_1;q,u_2)}\right],\\
		M_k^*(u_1,u_2)&=\frac{1}{v_m}\int\on{d}q \left[1-e^{-\beta u(0,u_1;q,u_2)}\right]q_z^k.
	\end{split}
\end{equation*}

Note that in fact to identify a FCh model, it is necessary to designate at least three parameters: $(q,u,\phi)$, in which $q$ denotes the coordinate of its center of mass, $u$ denotes its orientation, and $\phi$ denotes the internal azimuthal angle. So the integral in $A^*,M_k^*$ has to be slightly modified and made more precise:
\red{这个A的定义有点问题}
\begin{equation}
	\begin{split}
		A^*(u_1,u_2)&=\frac{1}{v_m}\int_V\on{d}q \int_0^{2\pi}\on{d}\phi_1\int_0^{2\pi}\on{d}\phi_2 \left[1-e^{-\beta U_{att}(0,u_1,\phi_1;q,u_2,\phi_2)}\right],\\
		M_k^*(u_1,u_2)&=\frac{1}{v_m}\int_V\on{d}q \int_0^{2\pi}\on{d}\phi_1\int_0^{2\pi}\on{d}\phi_2 \left[1-e^{-\beta u(0,u_1, \phi_1; q,u_2, \phi_2)}\right]q_z^k.
	\end{split}
\end{equation}

In the expression $V$ is taken to be the whole space in which MC process will take place, and let the length scale on one dimension to be two times of a single molecule, i.e. denote $L = 2m\sigma_b$.

Since the two integrals are of the same type, we only illustrate the computation for $M_k^*$. By using cylindrical coordinates, write $\on{d}q=r\on{d}r\on{d}z\on{d}\phi$. Then the expression of $M_k^*$ becomes
\begin{equation}
	M_k^*(u_1,u_2)=\frac{2\pi}{v_m}\int_{-L/2}^{L/2}\on{d}z\int_{-L/2}^{L/2}\on{d}r \int_0^{2\pi}\on{d}\phi_1\int_0^{2\pi}\on{d}\phi_2 \left[1-e^{-\beta u(0,u_1, \phi_1; q,u_2, \phi_2)}\right]z^kr.
\end{equation}

Except for using Monte Carlo method to estimate the preceding integral, we can also use an averaged method by dividing $z,r,\phi_1,\phi_2$ into uniform pieces. In practice, $z,r$ are always taken to be divided into $50$ pieces and $\phi_1,\phi_2$ are divided into $10$ pieces.

\subsubsection{Collision Detection}
In the estimation of the integral introduced above, there are many pairs of configurations which made $u(0,u_1,\phi_1;q,u_2,\phi_2)=0$ and hence contributes nothing to the whole integral, since the interaction between FCh molecules are relatively short-ranged. So we can calculate the shortest distance between two molecules and judge if it is within the cutoff distance to optimize the integral computation. We give the algorithm here without proof, since it is easy with elementary vector analysis.

\begin{algorithm}[H]
	\caption{Shortest Distance between a Point and a Segment}
	\label{Alg:DisPointSeg}
	\KwData{point $\vec{p}$, start point of the segment $\vec{s}$, end point of the segment $\vec{t}$}
	\KwResult{shortest distance $d$ between $\vec{p}$ and the segment defined by $\vec{s},\vec{t}$}
	$\vec{u}\leftarrow\dfrac{\vec{t}-\vec{s}}{\|\vec{t}-\vec{s}\|_2}$\;
	$\vec{ps}\leftarrow\vec{p}-\vec{s}$, $\vec{pt}\leftarrow\vec{p}-\vec{t}$\;
	\uIf{$\vec{u}\cdot\vec{ps}<0$}{
		$d\leftarrow\|\vec{ps}\|_2$\;
	}\uElseIf{$\vec{u}\cdot\vec{pt}>0$}{
		$d\leftarrow\|\vec{pt}\|_2$\;
	}\Else{
		$d\leftarrow\|\vec{u}\times\vec{ps}\|_2$\;
	}
\end{algorithm}

\begin{algorithm}[H]
	\caption{Shortest Distance between Two Segments}
	\label{Alg:DisSegSeg}
	\KwData{length, center and unit orientation vector of the segment $l_i, \vec{r}_i, \vec{u}_i$}
	\KwResult{shortest distance $d$ between the two segment}
	$\vec{s}\leftarrow\vec{r}_1-\vec{r}_2$\;
	$f\leftarrow$True\;
	\uIf{$1-(\vec{u}_1\cdot\vec{u}_2)^2=0$}{
		$f\leftarrow$False\tcp*[l]{Parallel segments}
	}\Else{
		$\lambda_1\leftarrow\dfrac{1}{1-(\vec{u}_1\cdot\vec{u}_2)^2}\left(-\vec{s}\cdot\vec{u}_1+(\vec{s}\cdot\vec{u}_2)(\vec{u}_1\cdot\vec{u}_2)\right)$\;
		$\lambda_2\leftarrow\dfrac{1}{1-(\vec{u}_1\cdot\vec{u}_2)^2}\left(\vec{s}\cdot\vec{u}_2-(\vec{s}\cdot\vec{u}_1)(\vec{u}_1\cdot\vec{u}_2)\right)$\;
		
		\eIf{$\left(|\lambda_1| > l_1/2\right) \vee \left(|\lambda_2|>l_2/2\right)$}{
			$f\leftarrow$False\tcp*[l]{Skew segments}
		}{
			$f\leftarrow$True\tcp*[l]{Segments intesects}
		}
	}
	
	\eIf{$f$}{
		$d\leftarrow\|\vec{s}+\lambda_1\vec{u}_1-\lambda_2\vec{u}_2\|_2$
	}{
		$d$ is taken to be the minimum of four shortest distance from four endpoints to the other segments using Alg. \ref{Alg:DisPointSeg}.
	}
\end{algorithm}

Using the algorithms introduced above, the efficiency of Monte Carlo estimation of the integral can be largely improved.

\subsection{DFT calculation}
With all the data coming from the Monte Carlo method, using iteration to find out the corresponding ODF with a given packing fraction $\eta$ can be directly performed, which has been already illustrated in Sec. \ref{Sec:DFT} 

\section{Code Calibration: Hard Spherocylinders (HSC) Model}
To verify the correctness of the computation program, we used the data from Jackson in 1995[REF]. In the paper the author re-examined the phase diagram of hard spherocylinders by Monte-Carlo simulation in a canonical ensemble. We used our theoretical method to calculate the phase diagram of HSC, and the result is in well accordance with Jackson's simulation results.

The paper calculated HSC with four different ratio of length and diameter $\kappa=3,3.2,4,5$, and the nematic-isotropic transition point is compared in the next table.

\begin{table}[h!]
	\caption{Comparison of transition point}
	\label{tab:comparison1995}
	\centering
	\begin{tabular}{c|cccc}
		\toprule
		 & $\kappa=3$ & $\kappa=3.2$ & $\kappa=4$ & $\kappa=5$\\
		\midrule
		$\eta_{MC}$ &  $0.577$ & $0.513$ & $0.472$ & $0.408$\\
		$\eta_{DFT}$ & $0.530$ & $0.515$ & $0.460$ & $0.405$\\
		\bottomrule
	\end{tabular}
\end{table}

It can be seen from Tab. \ref{tab:comparison1995} that the result given by DFT method is in accordance with the result given by Monte Carlo method in a canonical ensemble. The more detailed data of pressure is also in well accordance, see the following figure.
\red{Lack several Pics}

\section{The Isotropic-Nematic Phase Transition}



\section{The Cholesteric Pitch}

\section{Comparison with Simulation Results}

\section{Other Trials }

