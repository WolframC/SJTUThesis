% !TEX root = ../thesis.tex

\chapter{Introduction}


This undergraduate thesis mainly investigated electrically neutral lyotropic liquid crystals, this is a kind of typical soft matter. Historically, liquid crystals were discovered in 1888 when the Australian botanist Friedrich Reinitzer studied cholesterol benzoate, which is now known as cholesteric liquid crystals, and his friend German physicist Otto Lehmann named it "Liquid Crystal".

Microscopically, liquid crystals are mostly rod-like molecules, which enables them to behave anisotropically under certain circumstances. Under high temperature (so that the kinetic energy dominates) or very low density, molecules can still move freely regardless of its rod-like shape, and they behaves just like normal fluids, this is the isotropic phase. But it can be imagined that under high density or low temperature (so that the interaction between molecules will dominate over kinetic energy), molecules are pushing each other and tend to lean on each other in a near parallel way, so they will be oriented preferably in a certain direction, this is the nematic phase. And further if they are more than rod-like, having chiral mutual interactions between each other, they further tend to lean on each other with a preferable non zero angle, behaves periodically in the space, this is the cholesteric phase.

But the understanding of liquid crystal at a molecular level is still challenging, in a preceding work by Liang\cite{Liang2017SM}, a coarse-grained molecular model is developed and represented by flexible chain with helical interactions (FCh). This is a reasonable approximation to a number of ordinary cholesteric liquid crystals, such as double strand DNA. Both nematic phase and cholesteric phase was observed by molecular dynamics (MD) simulation FCh molecules.

To further investigate how the phase transition happens and provide insight into the relationship between microscopic molecular characteristics and the macroscopic phase behavior, a theoretical method is developed in this thesis to give a prediction of thermodynamic properties of FCh model. The theory is based on Onsager's original theory\cite{Onsager1949NYAS} of phase transition of liquid crystals, and makes use of DFT method to give a depiction of the thermodynamic properties of the phase transition of FCh. In addition, a modular program is developed from scratch for the phase diagram computation of a wide range of liquid crystal molecules including FCh.

