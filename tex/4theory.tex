% !TeX root = ../thesis.tex
\newcommand{\mat}[1]{\left(\begin{matrix}#1\end{matrix}\right)}
\newcommand{\arr}[2][n]{#2_1,#2_2,\cdots,#2_{#1}}
\newcommand{\xsim}[1]{\stackrel{#1}{\sim}}
\newcommand{\iprod}[2]{\langle#1,#2\rangle}
\newcommand{\mbb}[1]{\mathbb{#1}}
\newcommand{\mbf}[1]{\mathbf{#1}}
\newcommand{\mcal}[1]{\mathcal{#1}}
\newcommand{\mfk}[1]{\mathfrak{#1}}
\newcommand{\mrm}[1]{\mathrm{#1}}
\newcommand{\mcr}[1]{\mathscr{#1}}
\newcommand{\on}[1]{\operatorname{#1}}
\newcommand{\ol}[1]{\overline{#1}}
\newcommand{\wt}[1]{\widetilde{#1}}
\newcommand{\mr}[1]{\mathring{#1}}
\newcommand{\lr}[1]{\langle{#1}\rangle}
\newcommand{\red}[1]{\textcolor{red}{#1}}
\newcommand{\blue}[1]{\textcolor{blue}{#1}}
\newcommand{\green}[1]{\textcolor{green}{#1}}
\newcommand{\tbf}[1]{\textbf{#1}}
\newcommand{\tit}[1]{\textit{#1}}

\chapter{Theoretical Framework}

\section{Statistical Mechanics}
\subsection{Theory of Ensembles}

\subsection{Liquid Crystal Theory of Onsager}

\subsection{Parsons-Lee's Approximation}

\subsection{Straley's Method}

\section{Density Function Theory}
\subsection{Variation Method}

\subsection{Iteration Equation}

\section{Application to FCh Model}
This section collects the concrete computation of properties of FCh model.


First, write
\begin{equation}
	F^*=F_{id}^*+F_{ex}^*,
\end{equation}
the first term is
\begin{equation}
	F_{id}^*=\ln\eta-1+\int\on{d}\vec{u}f(\vec{u})\ln f(\vec{u}),
\end{equation}
then due to Straley[], we have
\begin{align}
	F_{ex}^*&=\frac{\eta}{2}\int\on{d}\vec{u}_1\on{d}\vec{u}_2M_0^*(\vec{u}_1,\vec{u}_2)f(\vec{u}_1)f(\vec{u}_2)-K_t^*q+\frac{1}{2}K_2^*q^2\\
	K_t^*&=-\frac{\eta}{2}\int\on{d}\vec{u}_1\on{d}\vec{u}_2M_1^*(\vec{u}_1,\vec{u}_2)f(\vec{u}_1)f'(\vec{u}_2)u_{2y}\\
	K_2^*&=-\frac{\eta}{2}\int\on{d}\vec{u}_1\on{d}\vec{u}_2M_2^*(\vec{u}_1,\vec{u}_2)f'(\vec{u}_1)f'(\vec{u}_2)u_{1y}u_{2y}\\
\end{align}
and in which the reduced $k$-th moment is given by
\begin{equation}
	M_k^*(\vec{u}_1,\vec{u}_2)=\frac{1}{v_m}\int\on{d}\vec{r}_c\left[e^{-\beta U(\vec{r}_c,\vec{u}_1,\vec{u}_2)}-1\right]r_{cz}^k
\end{equation}

In addition, the first term of $F_ex^*$ is divided into two parts(due to Parsons-Lee)
\begin{equation}
	\begin{split}
	\frac{\eta}{2}\int\on{d}\vec{u}_1\on{d}\vec{u}_2M_0^*(\vec{u}_1,\vec{u}_2)f(\vec{u}_1)f(\vec{u}_2)&=G(\eta)V_{int}\\
	&+\frac{\eta}{2}A_{int}
	\end{split}
\end{equation}

in which $G(\eta)=\dfrac{4\eta-3\eta^2}{8(1-\eta)^2}$, and
\begin{align}
V^*(\vec{u}_1,\vec{u}_2)=\frac{1}{v_m}\int\on{d}\vec{r}_c\left[e^{-\beta U_{rep}(\vec{r}_c,\vec{u}_1,\vec{u}_2)}-1\right],\\
V_{int} = \int\on{d}\vec{u}_1\on{d}\vec{u}_2V^*(\vec{u}_1,\vec{u}_2)f(\vec{u}_1)f(\vec{u}_2), \\
A^*(\vec{u}_1,\vec{u}_2)=\frac{1}{v_m}\int\on{d}\vec{r}_c\left[e^{-\beta U_{att}(\vec{r}_c,\vec{u}_1,\vec{u}_2)}-1\right],\\
A_{int}=\int\on{d}\vec{u}_1\on{d}\vec{u}_2A^*(\vec{u}_1,\vec{u}_2)f(\vec{u}_1)f(\vec{u}_2).
\end{align}

Up to now all the required quantities are presented. In the next section we write down explicitly all the important physical properties.

\section{Properties}
\subsection{Reduced free energy}
\begin{equation}
	\begin{split}
	F^*&=\ln\eta -1+\int\on{d}\vec{u}f(\vec{u})\ln f(\vec{u})+\frac{4\eta-3\eta^2}{8(1-\eta)^2}V_{int}+\frac{\eta}{2}A_{int}-K_t^*q+\frac{1}{2}K_2^*q^2
	\end{split}
\end{equation}

\subsection{Reduced Pressure}
The reduced pressure $P^*=\beta v_m P=\eta^2\dfrac{\partial F^*}{\partial\eta}$
\begin{equation}
	P^* = \eta + \frac{\eta^2(2-\eta)}{4(1-\eta)^3}V_{int}+\frac{1}{2}\eta^2A_{int}-\frac{1}{2}\eta K_t^*q
\end{equation}

\subsection{Chemical Potential}
The reduced chemical potential $\mu^*=\beta\mu=\eta\dfrac{\partial F^*}{\partial\eta}+F^*$
\begin{equation}
	\mu^*=\ln\eta+\int\on{d}\vec{u}f(\vec{u})\ln f(\vec{u})+\frac{\eta(3\eta^2-9\eta+8)}{8(1-\eta)^3}V_{int}+\eta A_{int}-K_t^*q
\end{equation}

\section{DFT part}
This part deals with the calculation of ODF by minimizing the free energy without cholesteric terms, i.e. in the pure nematic phase
\begin{equation}
	F_{n}^*=\ln\eta -1+\int\on{d}\vec{u}f(\vec{u})\ln f(\vec{u})+G(\eta)V_{int}+\frac{\eta}{2}A_{int}
\end{equation}
Now we need to minimize the above expression with respect to $\int f(\vec{u})\on{d}\vec{u}=1$, so consider the variation with respect to $f$, for convenience, take $f_{\epsilon}=f+\epsilon g$ then we require
\begin{equation}
	\frac{\partial}{\partial\epsilon}\left[F_{n}^*(f_{\epsilon})+\lambda\left(\int f_{\epsilon}\on{d}\vec{u}-1\right)\right]|_{\epsilon=0}=0, \forall g
\end{equation}
By calculation, this gives
\begin{equation}
	\int\on{d}\vec{u}_1g(\vec{u}_1)\left(1+\ln f(\vec{u}_1)+2G(\eta)\int\on{d}\vec{u}_2f(\vec{u}_2)V^*(\vec{u}_1,\vec{u}_2))+\eta\int\on{d}\vec{u}_2f(\vec{u}_2)A^*(\vec{u}_1,\vec{u}_2))+\lambda\right)=0
\end{equation}
Hence
\begin{equation}
	1+\lambda+\ln f(\vec{u}_1)+2G(\eta)\int\on{d}\vec{u}_2f(\vec{u}_2)V^*(\vec{u}_1,\vec{u}_2))+\eta\int\on{d}\vec{u}_2f(\vec{u}_2)A^*(\vec{u}_1,\vec{u}_2))=0
\end{equation}
By writing
\begin{equation}
	\begin{split}
		I^{rep}(\vec{u})&=2G(\eta)\int\on{d}\vec{u}_2f(\vec{u}_2)V^*(\vec{u},\vec{u}_2)\\
		I^{att}(\vec{u})&=\eta\int\on{d}\vec{u}_2f(\vec{u}_2)A^*(\vec{u},\vec{u}_2))
	\end{split}
\end{equation}

The iteration equation is given by
\begin{equation}
	f_{n+1}(\vec{u})=\on{Norm}\left(e^{-I_n^{rep}(\vec{u})-I_n^{att}(\vec{u})}\right)
\end{equation}
in which $\on{Norm}$ means normalizing the integral over all $\vec{u}$ to $1$.

