% 设置 biblatex 额外选项
% \PassOptionsToPackage{gbpub=false, gbtype=false}{biblatex}

% 载入 SJTUThesis 模版
% \documentclass[degree=doctor, zihao=-4, language=english, review]{sjtuthesis}
%\documentclass[degree=master, zihao=-4]{sjtuthesis}
 \documentclass[degree=bachelor, language=english, openany, oneside]{sjtuthesis}
% \documentclass[degree=course, language=english, openright, twoside]{sjtuthesis}
% 选项
%   degree=[doctor|master|bachelor|course],     % 必选,学位类型
%   language=[chinese|english],                 % 可选(默认:chinese),论文的主要语言
%   bibstyle=[gb7714-2015|gb7714-2015ay|ieee],  % 可选(默认:gb7714-2015),参考文献样式
%   review,                                     % 可选(默认:关闭),盲审模式

% 所有其它可能用到的包都统一放到这里了,可以根据自己的实际添加或者删除。
\usepackage{sjtuthesis}

% 定义图片文件目录与扩展名
\graphicspath{{figure/}}
\DeclareGraphicsExtensions{.pdf,.eps,.png,.jpg,.jpeg}

% 导入参考文献数据库
\addbibresource{bib/thesis.bib}

% 信息录入,必须在导言区进行!
% !TEX root = ../thesis.tex

%TC:ignore

\title{胆甾液晶的密度泛函理论研究}
\author{文\quad{}聪}
\studentid{516111910163}
\supervisor{吴量,孙淮}
% \assisupervisor{某某教授}
\degree{理学学士}
\major{化学(致远荣誉计划)}
\department{化学化工学院,致远学院}
\coursename{某某课程}
\date{2020年05月10日}
% \fund{国家 973 项目 (No. 2025CB000000) \\ 国家自然科学基金 (No. 81120250000)}
\keywords{液晶,统计力学,密度泛函理论}

\entitle{Density Function Theory of Cholesteric Liquids}
\enauthor{Cong Wen}
\ensupervisor{Liang Wu, Huai Sun}
% \enassisupervisor{Prof. Uom Uom}
\endegree{Bachelor of Science}
\enmajor{Chemistry (Zhiyuan Honors Program)}
\endepartment{School of Chemistry and Chemical Engineering, Zhiyuan College}
\endate{May. 10th, 2020}
% \enfund{National Basic Research Program of China (Grant No. 2025CB000000) \\
%   National Natural Science Foundation of China (Grant No. 81120250000)}
\enkeywords{Liquid Crystal, Statistical Mechanics, Density Function Theory}

%TC:endignore


% 自定义项目标签名称
% \sjtuSetLabel{
%   listfigure = {图\quad 录},
%   listtable  = {表\quad 录}
% }

\begin{document}

% 无编号内容:中英文论文封面、授权页
\maketitle
\makeDeclareOriginality[pdf/originality.pdf]
\makeDeclareAuthorization

% 使用罗马数字对前言编号
\frontmatter

% 摘要
% !TEX root = ../thesis.tex

\begin{abstract}
	对于一类特殊的分子,在计算机分子模拟中观察到了其胆甾相(手性向列向),但整个计算过程仍然非常耗时,所以我们构建了一个可以更加便捷地预测这类分子热力学性质的理论框架。理论体系以Onsager的液晶相变理论为基础,以Parsons-Lee的硬球近似方法为改进,我们可以用密度泛函理论(DFT)计算出这类分子的径向分布函数(ODF)。为了预测其胆甾相,我们进一步使用Straley的近似方法,以向列相的ODF近似为胆甾相的局部ODF,从而可以预测得到液晶分子在胆甾相下的热力学性质。
\end{abstract}

\begin{enabstract}
  Cholesteric phase is found for certain liquid crystals molecules in computer simulation, but the process is still very time-consuming, so we constructed a theoretical framework to predict the thermodynamic properties of this molecule. Based on Onsager’s theory and Parsons-Lee’s improvement of phase transition of liquid crystals, we used density function theory to calculate the orientational distribution function of the molecule. And in addition, by applying Straley’s approximation method, we calculated the cholesteric pitches for a given density. We thus found a way to predict the thermodynamic properties of cholesteric liquid crystals.
\end{enabstract}


% 目录、插图目录、表格目录
\tableofcontents
\listoffigures
\listoftables
\listofalgorithms

% 主要符号、缩略词对照表
% !TEX root = ../thesis.tex

%TC:ignore

\begin{nomenclature}{rl}
\label{chap:symb}
  $\epsilon_0$ & Energy Unit \\
  $k$ & Boltzmann Constant \\
  $T$ & Temperature \\
  $\beta$ & $\frac{1}{kT}$ \\
  $O$ & Ensemble Average of a Physical Property \\
  $O^*$ & Reduced Ensemble Average of a Physical Property \\
  $F$ & Free Energy \\
  $P$ & Pressure \\
  $\mu$ & Chemical Potential \\
  $p$ & Cholesteric Pitch \\
  $q$ & $2\pi/p$, Cholesteric Pitch Wave Vector \\
  $f(\vec{u})$ & Orientational Distribution Function(ODF) \\
  $v_m$ & Molecular Volume \\
  $\rho$ & Number Density \\
  $\eta$ & $\rho v_m$, Packing Fraction\\
\end{nomenclature}

%TC:endignore


% 使用阿拉伯数字对正文编号
\mainmatter

% 论文正文
% !TeX root = ../thesis.tex

\chapter{Molecular Model}

\section{Flexible Chain with Helical Interactions (FCh)}

\section{Existing Simulation Results}
% !TeX root = ../thesis.tex
\newcommand{\mat}[1]{\left(\begin{matrix}#1\end{matrix}\right)}
\newcommand{\arr}[2][n]{#2_1,#2_2,\cdots,#2_{#1}}
\newcommand{\xsim}[1]{\stackrel{#1}{\sim}}
\newcommand{\iprod}[2]{\langle#1,#2\rangle}
\newcommand{\mbb}[1]{\mathbb{#1}}
\newcommand{\mbf}[1]{\mathbf{#1}}
\newcommand{\mcal}[1]{\mathcal{#1}}
\newcommand{\mfk}[1]{\mathfrak{#1}}
\newcommand{\mrm}[1]{\mathrm{#1}}
\newcommand{\mcr}[1]{\mathscr{#1}}
\newcommand{\on}[1]{\operatorname{#1}}
\newcommand{\ol}[1]{\overline{#1}}
\newcommand{\wt}[1]{\widetilde{#1}}
\newcommand{\mr}[1]{\mathring{#1}}
\newcommand{\lr}[1]{\langle{#1}\rangle}
\newcommand{\red}[1]{\textcolor{red}{#1}}
\newcommand{\blue}[1]{\textcolor{blue}{#1}}
\newcommand{\green}[1]{\textcolor{green}{#1}}
\newcommand{\tbf}[1]{\textbf{#1}}
\newcommand{\tit}[1]{\textit{#1}}

\chapter{Theoretical Framework}

\section{Statistical Mechanics}
\subsection{Theory of Ensembles}

\subsection{Liquid Crystal Theory of Onsager}

\subsection{Parsons-Lee's Approximation}

\subsection{Straley's Method}

\section{Density Function Theory}
\subsection{Variation Method}

\subsection{Iteration Equation}

\section{Application to FCh Model}
This section collects the concrete computation of properties of FCh model.


First, write
\begin{equation}
	F^*=F_{id}^*+F_{ex}^*,
\end{equation}
the first term is
\begin{equation}
	F_{id}^*=\ln\eta-1+\int\on{d}\vec{u}f(\vec{u})\ln f(\vec{u}),
\end{equation}
then due to Straley[], we have
\begin{align}
	F_{ex}^*&=\frac{\eta}{2}\int\on{d}\vec{u}_1\on{d}\vec{u}_2M_0^*(\vec{u}_1,\vec{u}_2)f(\vec{u}_1)f(\vec{u}_2)-K_t^*q+\frac{1}{2}K_2^*q^2\\
	K_t^*&=-\frac{\eta}{2}\int\on{d}\vec{u}_1\on{d}\vec{u}_2M_1^*(\vec{u}_1,\vec{u}_2)f(\vec{u}_1)f'(\vec{u}_2)u_{2y}\\
	K_2^*&=-\frac{\eta}{2}\int\on{d}\vec{u}_1\on{d}\vec{u}_2M_2^*(\vec{u}_1,\vec{u}_2)f'(\vec{u}_1)f'(\vec{u}_2)u_{1y}u_{2y}\\
\end{align}
and in which the reduced $k$-th moment is given by
\begin{equation}
	M_k^*(\vec{u}_1,\vec{u}_2)=\frac{1}{v_m}\int\on{d}\vec{r}_c\left[e^{-\beta U(\vec{r}_c,\vec{u}_1,\vec{u}_2)}-1\right]r_{cz}^k
\end{equation}

In addition, the first term of $F_ex^*$ is divided into two parts(due to Parsons-Lee)
\begin{equation}
	\begin{split}
	\frac{\eta}{2}\int\on{d}\vec{u}_1\on{d}\vec{u}_2M_0^*(\vec{u}_1,\vec{u}_2)f(\vec{u}_1)f(\vec{u}_2)&=G(\eta)V_{int}\\
	&+\frac{\eta}{2}A_{int}
	\end{split}
\end{equation}

in which $G(\eta)=\dfrac{4\eta-3\eta^2}{8(1-\eta)^2}$, and
\begin{align}
V^*(\vec{u}_1,\vec{u}_2)=\frac{1}{v_m}\int\on{d}\vec{r}_c\left[e^{-\beta U_{rep}(\vec{r}_c,\vec{u}_1,\vec{u}_2)}-1\right],\\
V_{int} = \int\on{d}\vec{u}_1\on{d}\vec{u}_2V^*(\vec{u}_1,\vec{u}_2)f(\vec{u}_1)f(\vec{u}_2), \\
A^*(\vec{u}_1,\vec{u}_2)=\frac{1}{v_m}\int\on{d}\vec{r}_c\left[e^{-\beta U_{att}(\vec{r}_c,\vec{u}_1,\vec{u}_2)}-1\right],\\
A_{int}=\int\on{d}\vec{u}_1\on{d}\vec{u}_2A^*(\vec{u}_1,\vec{u}_2)f(\vec{u}_1)f(\vec{u}_2).
\end{align}

Up to now all the required quantities are presented. In the next section we write down explicitly all the important physical properties.

\section{Properties}
\subsection{Reduced free energy}
\begin{equation}
	\begin{split}
	F^*&=\ln\eta -1+\int\on{d}\vec{u}f(\vec{u})\ln f(\vec{u})+\frac{4\eta-3\eta^2}{8(1-\eta)^2}V_{int}+\frac{\eta}{2}A_{int}-K_t^*q+\frac{1}{2}K_2^*q^2
	\end{split}
\end{equation}

\subsection{Reduced Pressure}
The reduced pressure $P^*=\beta v_m P=\eta^2\dfrac{\partial F^*}{\partial\eta}$
\begin{equation}
	P^* = \eta + \frac{\eta^2(2-\eta)}{4(1-\eta)^3}V_{int}+\frac{1}{2}\eta^2A_{int}-\frac{1}{2}\eta K_t^*q
\end{equation}

\subsection{Chemical Potential}
The reduced chemical potential $\mu^*=\beta\mu=\eta\dfrac{\partial F^*}{\partial\eta}+F^*$
\begin{equation}
	\mu^*=\ln\eta+\int\on{d}\vec{u}f(\vec{u})\ln f(\vec{u})+\frac{\eta(3\eta^2-9\eta+8)}{8(1-\eta)^3}V_{int}+\eta A_{int}-K_t^*q
\end{equation}

\section{DFT part}
This part deals with the calculation of ODF by minimizing the free energy without cholesteric terms, i.e. in the pure nematic phase
\begin{equation}
	F_{n}^*=\ln\eta -1+\int\on{d}\vec{u}f(\vec{u})\ln f(\vec{u})+G(\eta)V_{int}+\frac{\eta}{2}A_{int}
\end{equation}
Now we need to minimize the above expression with respect to $\int f(\vec{u})\on{d}\vec{u}=1$, so consider the variation with respect to $f$, for convenience, take $f_{\epsilon}=f+\epsilon g$ then we require
\begin{equation}
	\frac{\partial}{\partial\epsilon}\left[F_{n}^*(f_{\epsilon})+\lambda\left(\int f_{\epsilon}\on{d}\vec{u}-1\right)\right]|_{\epsilon=0}=0, \forall g
\end{equation}
By calculation, this gives
\begin{equation}
	\int\on{d}\vec{u}_1g(\vec{u}_1)\left(1+\ln f(\vec{u}_1)+2G(\eta)\int\on{d}\vec{u}_2f(\vec{u}_2)V^*(\vec{u}_1,\vec{u}_2))+\eta\int\on{d}\vec{u}_2f(\vec{u}_2)A^*(\vec{u}_1,\vec{u}_2))+\lambda\right)=0
\end{equation}
Hence
\begin{equation}
	1+\lambda+\ln f(\vec{u}_1)+2G(\eta)\int\on{d}\vec{u}_2f(\vec{u}_2)V^*(\vec{u}_1,\vec{u}_2))+\eta\int\on{d}\vec{u}_2f(\vec{u}_2)A^*(\vec{u}_1,\vec{u}_2))=0
\end{equation}
By writing
\begin{equation}
	\begin{split}
		I^{rep}(\vec{u})&=2G(\eta)\int\on{d}\vec{u}_2f(\vec{u}_2)V^*(\vec{u},\vec{u}_2)\\
		I^{att}(\vec{u})&=\eta\int\on{d}\vec{u}_2f(\vec{u}_2)A^*(\vec{u},\vec{u}_2))
	\end{split}
\end{equation}

The iteration equation is given by
\begin{equation}
	f_{n+1}(\vec{u})=\on{Norm}\left(e^{-I_n^{rep}(\vec{u})-I_n^{att}(\vec{u})}\right)
\end{equation}
in which $\on{Norm}$ means normalizing the integral over all $\vec{u}$ to $1$.


% !TeX root = ../thesis.tex

\chapter{Results and Discussion}

\section{The Isotropic-Nematic Phase Transition}

\section{The Cholesteric Pitch}

\section{Comparison with Simulation Results}



% 全文总结
\input{tex/summary}

% 使用英文字母对附录编号
\appendix

% 附录内容,本科学位论文可以用翻译的文献替代。
%\input{tex/app_maxwell_equations}
%\input{tex/app_flow_chart}

% 文后无编号部分
\backmatter

% 参考资料
\printbibliography[heading=bibintoc]

% 用于盲审的论文需隐去致谢、发表论文、参与项目、申请专利、简历

% 致谢
% !TEX root = ../thesis.tex

%TC:ignore

\begin{acknowledgements}
	First of all, I am very grateful for my advisor Liang Wu and Huai Sun, they never ceases to encourage me and tolerate my faults, they helped me a lot in both study and life. I am also thankful for my parents, who gave me life and raised me up, and most importantly, they are always supporting me to find what I really need. At last, I would thank those who encouraged and trusted me in the past four years, particularly my friends and classmates.
\end{acknowledgements}

%TC:endignore


% 发表论文、参与项目、申请专利、简历
% 盲审论文中,发表学术论文及参与科研情况等仅以第几作者注明即可,不要出现作者或他人姓名

%\input{tex/publications}
%\input{tex/projects}
%\input{tex/patents}
% !TEX root = ../thesis.tex

%TC:ignore

\begin{resume}
  \subsection*{Fundamental Information}
    WEN, Cong, born on 12.01.1998

  \subsection*{教育背景}
  \begin{itemize}
    \item yyyy 年 mm 月至今,上海交通大学,博士研究生,xx 专业
    \item yyyy 年 mm 月至 yyyy 年 mm 月,上海交通大学,硕士研究生,xx 专业
    \item yyyy 年 mm 月至 yyyy 年 mm 月,上海交通大学,本科,xx 专业
  \end{itemize}

  \subsection*{联系方式}
  \begin{itemize}
    \item 地址: 上海市闵行区东川路 800 号,200240
    \item E-mail: \email{xxx@sjtu.edu.cn}
  \end{itemize}
\end{resume}

%TC:endignore


% 中文学士学位论文要求在最后有一个英文大摘要,单独编页码,英文学士学位论文不需要
%\input{tex/end_english_abstract}

\end{document}
